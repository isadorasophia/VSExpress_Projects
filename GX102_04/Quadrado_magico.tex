\documentclass {article}
\pagenumbering{gobble} % Remove page numbers (and reset to 1)

\usepackage[T1]{fontenc}
\usepackage[scaled]{beramono}

\usepackage[utf8]{inputenc}

\usepackage{color, soul}
\definecolor{bluekeywords}{rgb}{0.13,0.13,1}
\definecolor{greencomments}{rgb}{0,0.5,0}
\definecolor{redstrings}{rgb}{0.9,0,0}

\usepackage{listings}
\lstset{language=[Sharp]C,
showspaces=false,
showtabs=false,
breaklines=true,
showstringspaces=false,
breakatwhitespace=true,
escapeinside={(*@}{@*)},
commentstyle=\color{greencomments},
keywordstyle=\color{bluekeywords}\bfseries,
stringstyle=\color{redstrings},
basicstyle=\ttfamily
}

\usepackage {color,soul} % use colors!

\title {\bf{Quadrado Mágico}}
\date{\date{19 de Novembro, 2014}}
\begin{document}
	\maketitle
	\section{Solução}
	\subsection{Primeira solução, sem utilizar conceitos de OOP}
	
	\begin{lstlisting}
namespace Quadrado_Magico
{
    class Program
    {
        static void Main(string[] args)
        {
            int size, i, j, sum;
            int[,] magicSquare;

            bool isMagic = true;

            string line;
            string[] lineNumbers;

            size = Int32.Parse(Console.ReadLine());

            // Allocates the necessary memory
            lineNumbers = new string[size];
            magicSquare = new int[size, size];

            // Reads input
            (*@\hl{for (i = 0; i < size; i++)}@*)
            {
                line = Console.ReadLine();
                lineNumbers = line.Split(' ');

                (*@\hl{for (j = 0; j < size; j++)}@*)
                    magicSquare[i, j] = Int32.Parse(lineNumbers[j]);
            }

            // Takes the value of the sum of both diagonals, -1 if they are not the same
            sum = SumDiagonals(magicSquare, size);

            // If they weren't the same
            if (sum <= 0) 
                isMagic = false;

            // Checks if the sum of the columns diverge
            (*@\hl{for (i = 0; i < size \&\& isMagic; i++)}@*) 
                if (sum != SumColumn(magicSquare, i, size))
                    isMagic = false;
                    
            // Same for the lines
            (*@\hl{for (i = 0; i < size \&\& isMagic; i++)}@*) 
                if (sum != SumColumn(magicSquare, i, size))
                    isMagic = false;

            // Yay! Success!
            if (isMagic)
                Console.WriteLine(sum);
            // Nope
            else
                Console.WriteLine("-1");

            // Exit
            Console.ReadKey();
        }

        // Since it keeps asking for an object reference, all the methods are static... Returns the sum of the current column
        static int SumColumn(int[,] magicSquare, int column, int size)
        {
            int sum = 0, i;

            (*@\hl{for (i = 0; i < size; i++)}@*)
                sum += magicSquare[i, column];

            return sum;
        }

        // Returns the sum of the current line
        static int SumLine(int[,] magicSquare, int size, int line)
        {
            int sum = 0, j;

            (*@\hl{for (j = 0; j < size; j++)}@*)
                sum += magicSquare[line, j];

            return sum;
        }

        // Returns the sum of both diagonals, -1 if they are not the same
        static int SumDiagonals(int[,] magicSquare, int size)
        {
            int i, leftDiagonal = 0, rightDiagonal = 0;

            (*@\hl{for (i = 0; i < size; i++)}@*)
                leftDiagonal += magicSquare[i, i];

            (*@\hl{for (i = size - 1; i >= 0; i--)}@*)
                rightDiagonal += magicSquare[i, i];

            if (leftDiagonal == rightDiagonal)
                return leftDiagonal;
            else
                return -1;
        }
    }
}
    \end{lstlisting}
    
    \subsection{Segunda solução, utilizando conceitos de OOP}
    \begin{lstlisting}
namespace Quadrado_Magico
{
    class MagicSquare
    {
        private readonly int[,] square;
        private readonly int size;

        // Constructor
        public MagicSquare (int[,] numbersInput, int size)
        {
            int i, j;
            this.square = new int[size, size];

            (*@\hl{for (i = 0; i < size; i++)}@*)
                (*@\hl{for (j = 0; j < size; j++)}@*)
                    this.square[i, j] = numbersInput[i, j];

            this.size = size;
        }

        // Returns the sum of the current column
        public int SumColumn(int column)
        {
            int sum = 0, i;

            (*@\hl{for (i = 0; i < this.size; i++)}@*)
                sum += this.square[i, column];

            return sum;
        }

        // Returns the sum of the current line
        public int SumLine(int line)
        {
            int sum = 0, j;

            (*@\hl{for (j = 0; j < this.size; j++)}@*)
                sum += this.square[line, j];

            return sum;
        }

        // Returns the sum of both diagonals, -1 if they are not the same
        public int SumDiagonals()
        {
            int i, leftDiagonal = 0, rightDiagonal = 0;

            (*@\hl{for (i = 0; i < this.size; i++)}@*)
                leftDiagonal += this.square[i, i];

            (*@\hl{for (i = this.size - 1; i >= 0; i--)}@*)
                rightDiagonal += this.square[i, i];

            if (leftDiagonal == rightDiagonal)
                return leftDiagonal;
            else
                return -1;
        }
    }

    class Program
    {
        static void Main(string[] args)
        {
            int sum, size, i, j;
            int[,] magicSquare;

            MagicSquare amIPerf;

            bool isMagic = true;

            string line;
            string[] lineNumbers;

            size = Int32.Parse(Console.ReadLine());

            // Allocates the necessary memory
            lineNumbers = new string[size];
            magicSquare = new int[size, size];

            // Reads input
            (*@\hl{for (i = 0; i < size; i++)}@*)
            {
                line = Console.ReadLine();
                lineNumbers = line.Split(' ');

                (*@\hl{for (j = 0; j < size; j++)}@*)
                    magicSquare[i, j] = Int32.Parse(lineNumbers[j]);
            }

            // Creates the square instance
            amIPerf = new MagicSquare(magicSquare, size);

            // Takes the value of the sum of both diagonals, -1 if they are not the same
            sum = amIPerf.SumDiagonals();

            // If they weren't the same
            if (sum <= 0) 
                isMagic = false;

            // Checks if the sum of the columns diverge
            (*@\hl{for (i = 0; i < size \&\& isMagic; i++)}@*)
                if (sum != amIPerf.SumColumn(i))
                    isMagic = false;

            // Same for the lines
            (*@\hl{for (j = 0; j < size \&\& isMagic; j++)}@*)
                if (sum != amIPerf.SumLine(j))
                    isMagic = false;

            // Yay! Success!
            if (isMagic)
                Console.WriteLine(sum);
            // Nope
            else
                Console.WriteLine("-1");

            // Exit
            Console.ReadKey();
        }
    }
}
	\end{lstlisting}
\end{document}